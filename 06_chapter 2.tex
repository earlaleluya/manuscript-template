%%%%%%%%%%%%%%%%%%%%%%%%%%%%%%%%%%%%%%%%%%%%%%%%%%%%%%%%%%%%%%%%%%%%%%%
%---------------------------------------------------------------------%
% Start of CHAPTER 2 REVIEW OF RELATED LITERATURE
%---------------------------------------------------------------------%
%%%%%%%%%%%%%%%%%%%%%%%%%%%%%%%%%%%%%%%%%%%%%%%%%%%%%%%%%%%%%%%%%%%%%%%
     
\renewcommand{\thechapter}{\Roman{chapter}}
\chapter{Review of Related Literature}
    \thispagestyle{empty} 
    \renewcommand{\thechapter}{\arabic{chapter}}

    The review of related literature synthesizes and evaluates existing scholarly works relevant to the study's topic. It involves a comprehensive examination of peer-reviewed articles, books, and other academic sources, with the aim of identifying gaps, trends, and insights related to the research questions. This section not only provides a historical context for the study but also helps establish the significance of the research by demonstrating the current state of knowledge in the field. Researchers highlight key findings, methodologies, and theoretical frameworks from previous studies to contextualize their own work and justify its contribution to the existing body of knowledge. A well-structured review of related literature not only informs the reader about the existing research landscape but also lays the groundwork for developing a robust conceptual framework and research methodology.

    \textit{Important Note:} Please outline your RRL based on your objectives. Ask for guidance from your COE194 instructor, adviser, or panel member.

\section{Review for Objective/Phase 1}
    The.

\section{Review for Objective/Phase 2}
    The.

\section{Review for Objective/Phase 3}
    The.

\section{Summary}
    The summary should briefly synthesize the related studies. It aims to highlight the gaps and the novelty of the proposed solution.